% classy.tex

\chapter{Keeping It Classy}

Countless examples of structured programing in an object oriented language presented as proper technique exist both online and in programming texts. This is the equivalent of buying a set of golf clubs, but only ever using the seven iron.

Object orientation allows for the separation of concerns and facilitates refactoring. Two methodologies that allow changes to be introduced to software systems relatively easily.

Going forward, \href{http://www.javaworld.com/javaworld/jw-08-1999/jw-08-interfaces.html}{coding to interfaces} will be presented as the appropriate way to implement object oriented systems. This approach allows for the development of loosely coupled and highly cohesive systems. Coding for interfaces also allows the developer to prioritize development by factors like risk mitigation rather than requiring a class to wait on all it's components to be concretely implemented before starting work on that class. Coding to interfaces also allows for division of labor and collaboration. Perhaps most importantly, coding to interfaces allows objects to be tested in isolation.

