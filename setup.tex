%setup.tex

\chapter{Setting Up Your Programming Environment}

\section{DIY}
One click installers for all-in-one \glspl{IDE} can be nice, but they tend to encourage dependency on automata. They also promote, ``Well, if there 
isn't an installer for it, it can't be any good,'' thinking. It should be noted that often times creating an installer that works for Windows can be
prohibitively expensive for authors of tools who do not use Windows as their primary operating system or Visual Studio as their primary development 
environment.

\gls{FLOSS} software will be used in the examples hereafter; not because if its price-point, but rather because of the ability of open source software
to transfer knowledge to its end user. Want to know the best way to write a task for your build engine? Why not see how the authors of your build
engine did it? In order to write code, you must learn how to read code. By extension, to write code well, you must learn how to read code well. It is 
important to figure out why a piece of code was written the way it was and not just settle for duplicating and succesfully compiling it. Reading the 
code behind open source software is a cheap and easy way to learn from the masters. NAnt contains some of the most well written C\# code out there.

\gls{FLOSS} software can be a little more difficult to set up, but doing so is worth the extra effort. There are many freely available tutorials
online that will walk you through how to configure a software package for first-time use. In the process of setting up software, you will learn a
little bit more about your computer runs programs in addition to making your computer a little bit more your own.

The biggest reason for configuring your own environment is personal taste. You may prefer typing text on the command line in a program like 
Vim\footnote{http://www.vim.org/} over using an \gls{IDE}.

When it comes to programmer tools, one size definitely does not fit all. Something like ReSharper\footnote{http://www.jetbrains.com/resharper/} 
may give you unparalelled \gls{refactoring} and \gls{auto-completion} support, but starting Visual Studio to change a configuration setting is 
overkill.

\section{List of Tools}