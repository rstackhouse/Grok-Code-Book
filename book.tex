\documentclass{book}
\usepackage{hyperref}

\hypersetup{colorlinks=true,
	linkcolor=red}

% glossaries needs to come after hyperref if glossary entries are to be clickable
\usepackage[toc,acronym]{glossaries}

%Terms

\newglossaryentry{auto-completion}
{
name=auto-completion,
description={``A feature [of software programs] that suggests text automatically based on the first few characters that a user types.''\footnote{http://msdn.microsoft.com/en-us/library/dd921717(v=office.12).aspx}}
}

\newglossaryentry{refactoring}
{
name=refactoring,
description={``The process of modifying source code for the purpose of improving its readability and maintainability while retaining the program's functionality and behavior.'' \footnote{\url{http://developer.apple.com/library/ios/\#documentation/DeveloperTools/Conceptual/Xcode_Glossary/20-Glossary/Glossary.html}}}
}

%Acronyms

%use lower case for glossary entry
%IDE
\newglossaryentry{ide}
{
name=\glslink{IDE}{Integrated Development Environment},
text=Integrated Development Environment,
description={``An integrated development environment (IDE) (also known as integrated design environment, integrated debugging environment or interactive development environment) is a software application that provides comprehensive facilities to computer programmers for software development''\footnote{\url{http://en.wikipedia.org/wiki/Integrated_development_environment}}}
}

%use all-caps for acronym and in body text
\newacronym[description={\glslink{ide}{Integrated Development Environment}}]{IDE}{IDE}{Integrated Development Environment}

%FLOSS
% use \- to suggest hyphenation points
\newglossaryentry{floss}
{
name=\glslink{FLOSS}{Free/Libre/Open Source Software},
text=Free\/\-Libre\/\-Open Source Software,
description={``Free/\-Libre/\-Open-\-source software (FLOSS) is liberally licensed to grant the right of users to use, study, change, and improve its design through the availability of its source code'' \footnote{\url{http://en.wikipedia.org/wiki/Free_and_open_source_software}}}
}

\newacronym[description={\glslink{floss}{Free/\-Libre/\-Open Source Software}}]{FLOSS}{FLOSS}{Free/Libre/Open Source Software}

%NAnt
\newglossaryentry{nant}
{
name=\glslink{NAnt}{NAnt},
text=NAnt,
description={A \gls{FLOSS} build automation tool written in C\#\footnote{\url{http://nant.sourceforge.net/}}}
}

\newacronym[description={\glslink{nant}{Short for ``Not Ant''\footnote{\url{http://nant.sourceforge.net/release/latest/help/introduction/fog0000000042.html}}}}]{NAnt}{NAnt}{Not Ant}

%TDD
\newglossaryentry{tdd}
{
name=\glslink{TDD}{Test Driven Development},
text=Test Driven Development,
description={A software development methodology whereby the developer writes a failing test, writes code to make the test pass, and refactors the test as needed to prevent duplication of code and brittleness. The tests are automated with the use of an automated test harness.}
}

\newacronym[description={\glslink{tdd}{Test Driven Development}]{TDD}{TDD}{Test Driven Development}

%VCS
\newglossaryentry{vcs}
{
name=\glslink{VCS}{Version Control System},
text=Version Control System,
description={A software tool for managing changes made to source code. Make \gls{branching} and \gls{merging} possible.}
}

\newacronym[description={\glslink{vcs}{Version Control System}]{VCS}{VCS}{Version Control System}

%DVCS
\newglossaryentry{dvcs}
{
name=\glslink{DVCS}{Distributed Version Control System},
text=Distributed Version Control System,
description={A distributed form of \gls{vcs}. Whole repositories exist on the developer's machine rather than on a centralized server. Changes are sent between developers rather than client to server.}
}

\newacronym[description={\glslink{dvcs}{Distributed Version Control System}]{DVCS}{DVCS}{Distributed Version Control System}

%CI
\newglossaryentry{vcs}
{
name=\glslink{CI}{Continuous Integration},
text=Continuous Integration,
description={An automated process whereby source code is built and tested continuously, and reports on the outcome of the build/test process are sent to the developers.}
}

\newacronym[description={\glslink{vcs}{Continuous Integration}]{CI}{CI}{Continuous Integration}

\makeglossaries

\title{Grok Code Book}
\author{@GrokCodeBook \\ \url{http://twitter.com/\#!/GrokCodeBook}}

%defs.tex

\ifx \ifmonospace \undefined
	%http://www.opensource.apple.com/source/autoconf/autoconf-12/autoconf/config/texinfo.tex
	\def\ifmonospace{\ifdim\fontdimen3\font=0pt }
\fi

%http://stackoverflow.com/questions/2724760/how-to-write-c-in-latex/2728020#2728020
%C#
\def \CSharp {
\ifmonospace
    C\#
\else%
    %C\kern-.1667em\raise.30ex\hbox{\smaller{\#}}
    C\# %TODO: make this match example in url above
\fi
\spacefactor1000 }

\begin{document}
\maketitle
\addcontentsline{toc}{chapter}{Contents}
\pagenumbering{roman}
\tableofcontents

% The star suppresses numbering
\chapter*{Introduction}\normalsize
\addcontentsline{toc}{chapter}{Introduction}
\pagestyle{plain}

The solitary nature of much of computer programming and the number of different avenues that may lead to a career in programming make it 
difficult for best practices to permiate the field. A responsible author can no longer assume that a person seeking to learn about programming 
has any background in either computer science or mathematics. All jargon should be introduced in a glossary at minimum; regardless of how many 
works the author has authored. If a programming text is to build on some assumed knowledge, it must explicitly define said knowledge from the 
outset, and ought contain a ``Read First'' section. 

Also, the volume of information forthcoming about new technologies�frameworks, toolkits, tooling, etc.�drowns out information about programming 
well. By providing this text free of charge, holding ourselves to the highest standard of authorship and editorship, trying to make this text 
accessible to all that might read it, and acting as stewards and care-takers of this living document, we hope to raise the bar for published 
works pertaining to computer programming.

The examples included hereafter are executed in \CSharp and Microsoft.Net, but the concepts set forth therein are relevant to all OO languages and frameworks.

\pagestyle{headings}
\pagenumbering{arabic}

% Setting Up Your Programming Environment
%setup.tex

\chapter{Setting Up a Programming Environment}

\section{DIY}
One click installers for all-in-one \glspl{IDE} can be nice, but those installers and \glspl{IDE} tend to encourage dependency on automata. They also promote, ``Well, if there 
isn't an installer for it, it can't be any good,'' thinking. It should be noted that often times creating an installer that works for Windows can be prohibitively expensive for authors of tools who do not use Windows as their primary operating system or Visual Studio as their primary development environment.

\gls{FLOSS} software will be used in the examples hereafter; not because if its price-point, but rather because of the ability of open source software
to transfer knowledge to its end user. Want to know the best way to write a task for a build engine? Why not see how the authors of the build
engine did it? In order to write code, one must learn how to read code. By extension, to write code well, one must learn how to read code well. It is 
important to figure out why a piece of code was written the way it was and not just settle for duplicating and succesfully compiling it. Reading the 
code behind open source software is a cheap and easy way to learn from the masters. \gls{NAnt} contains some of the most well written \CSharp code out there.

\gls{FLOSS} software can be a little more difficult to set up, but doing so is worth the extra effort. There are many freely available tutorials
online that will walk the developer through how to configure a software package for first-time use. In the process of setting up software, developers learn a
little bit more about how a computer runs programs in addition to making their computers a little bit more their own.

The biggest reason for configuring a development environment as one sees fit is personal taste. Some developers prefer typing text on the command line in a program like 
Vim\footnote{\url{http://www.vim.org/}} over using an \gls{IDE} exclusively.

When it comes to programmer tools, one size definitely does not fit all. Something like ReSharper\footnote{\url{http://www.jetbrains.com/resharper/}} 
may afford developers unparalelled \gls{refactoring} and \gls{auto-completion} support, but starting Visual Studio to change a configuration setting or one 
line of code is overkill. One theme that will recur throughout the passages that follow is, 
``Use the right tool for the job at hand.''

\begin{quotation}
If all you have is a hammer, everything looks like a nail.

---Bernard Baruch
\end{quotation}

Choose tools wisely.

\section{Tooling}

\subsection{Build Automation}
Developers should always provide other developers on their team the ability to quickly build projects and test them (with automated tests as well as 
manually). Testing should always be a part of the review process as exercising the code can and does reveal errors that a visual inspection may 
miss. Build automation is the mechanism that extends developers this ability.

Tasks within a build automation system can be used to check style, run automated tests, and write reports as well as compiling source code.

Build automation is a critical to the practice of \gls{TDD}. If tests cannot be easily automated, they will not be run. Not out of developer laziness,
but for the reason of sheer number. As a codebase grows, the number of tests for it can easily number in the hundreds. If there isn't test harness for
automating those tests, they simply won't be run often enough.

\gls{NAnt} will be used in the examples that follow. MSBuild has some advantages in Windows-only shops---i.e. it is
installed by default on Windows Server machines. At the time of writing, it more of a hassle to run all the unit tests for a project from the command
-line in MSBuild than in NAnt. Also, there seems to be more of an air of acceptance towards customization---specifically writing one's own build 
tasks---in the NAnt community. 

While there is virtue in the philosophy, ``Why write what you can download for free?'' sometimes the needs of 
customization outweigh the benefits of using easily downloadable/purchased software. The philosophy set forth herein is, ``Weigh all options without bias, and make the
logical choice.'' Developers are---or at least always should be---part of a team. Sometimes, for the sake of progress, a developer needs to make
decisions that compromise on their own happiness. For instance, not using one's favorite build tool on a fresh project in a new job, because all the
other developers are invested in the use of another tool.

\subsection{Test Driven Development}

\gls{TDD} should be thought of as a developer's safety net. Trapeze artists use a safety net to keep them from falling to their deaths. The presence
of that net allows them to attempt things they might not otherwise. It allows them to overcome their fear to achieve greatness. \gls{TDD} is first 
and foremost about giving developers the confidence to improve their code.

\gls{NUnit} will be used in later examples as more support exists to automate it from the client machine.

\subsection{Version Control}
Things don't always go as planned. Sometimes defects will evade detection in testing. Having a way to quickly revert to a known working state is a 
luxury no developer can afford to go without. Version control can be a lot like insurance to the uninitiated: it's value isn't apparent until it's benefits have been experienced or woefully missed.

\glspl{VCS} also provide a facility called branching. \Gls{branching} permits experimentation within libraries. \Gls{merging} allows succesful 
experiments be kept. Otherwise, the branch can simply be abandoned. 

%TODO: find someone who knows something about this to write this bit. Dropbox?
\subsection{Backup}
To prevent lost work, source code and application configuration should be backed up regularly. \glspl{DVCS} have the advantage that each developer 
has a copy of the entire repository. Thus, as changes are passed back and forth between developers, backup occurs organically.

\subsection{Continuous Integration}
Based on the philosophy of making the hard things to do easier by doing them more often, ideally, continuous integration provides a means for 
keeping software in a ready-to-deploy state.

\gls{CI} systems wait for a trigger, and then perform a series of tasks once that trigger has been detected such as running unit tests,
emailing reports, updating documentation, and even performing version control tasks.

Common triggers are changes to the source code repository, elapsed time, and forced builds.

The \gls{CI} server has oft been called the heartbeat of the software project.

In the following pages, \gls{Buildbot} will be used as the CI server.

\section{A First Build}

\subsection{Get NAnt}
Download \gls{NAnt} from \url{http://sourceforge.net/projects/nant/files/nant/}. As of the time of this writing, it is recommended that one get
either the stable release of 0.90 or the nightly build. Unzip to C:$\backslash$Program Files$\backslash$nant-$\langle$version$\rangle$ or another centrally available 
location.

Modify the \em \hyperref[path]{\%PATH\%} \em environment variable to point to $\langle$NAnt-Installation-Directory$\rangle$$\backslash$bin (i.e. C:$\backslash$Program Files$\backslash$nant-0.90$\backslash$bin).

Test NAnt by \hyperref[launch-command-window]{launching a command window} and typing \em nant -help \em at the prompt. This should display NAnt's help text to the command window.

\subsection{Create a First Build File}
With the open command window, \hyperref[change-directories]{change directories} to the desktop.

\hyperref[create-directory]{Create a directory} named ``GrokCodeBook'' using the ``mkdir'' command.

Open a file for editing with a favorite \gls{text-editor} (a \hyperref[text-editors]{listing of text-editors} is provided in the \hyperref[tools]{Tools} appendix) and type the following:

%http://kb.mit.edu/confluence/display/ist/How+to+use+Tabs+in+LaTeX
\begin{tabbing}
$\langle$?xml version="1.0"?$\rangle$ \\
$\langle$pro\=ject name="Hello Grok Code Book" default="msg" basedir="."$\rangle$ \\
\>$\langle$description$\rangle$An example build file$\langle$/description$\rangle$ \\
\>$\langle$tar\=get name="msg"$\rangle$ \\
\>\>$\langle$echo message="Hello!"/$\rangle$ \\
\>$\langle$/target$\rangle$ \\
$\langle$/project$\rangle$
\end{tabbing}

Save the file as $\langle$Desktop$\rangle$$\backslash$GrokCodeBook$\backslash$default.build.

From the $\langle$Desktop$\rangle$$\backslash$GrokCodeBook directory, issue the command \em nant \em. This should result in the display of some banner text and the message ``Hello!'' on the command window.

\section{Getting It Under Control}
\subsection{Get Git}
Download msysgit from \url{http://code.google.com/p/msysgit/downloads/list}, and run the installer.

Launch Git Bash from the desktop icon or the start menu. A terminal window with ``MINGW32'' in the title bar should display on screen.

A welcome message should be displayed on the terminal window followed by the name of the machine followed by a tilde---a tilde (\~) is shorthand for the user's home directory in *NIX systems--- followed by the prompt (the \$ character).

\subsection{Making Introductions With Git}
At the prompt, type \em git config --global user.name "$\langle$Your Name Here$\rangle$" \em. The \em --global \em option instructs Git to store the supplied values in the \em \~/.gitconfig \em file versus the \em .gitconfig \em for a specific repository.

Next, type \em git config --get user.name \em. Whatever value that was specified for \em user.name \em should be displayed in the terminal window.

Now, supply Git with an email address by typing \em git config --global user.email $\langle$your-email@your-domain.com$\rangle$ \em. These detials will help other developer know who to contact should a problem arise (hopefully that eventuality will be avoided with judicious use of \gls{TDD}).

\section{A First Repository}
From the prompt, issue the command \em mkdir \~/Desktop/GrokCodeBook/code \em (when working with paths, you should use \hyperref[tab-completion]{tab completion} for speed). This will create the folder the examples will be contained in. Next, issue the command \em cd \~/Desktop/GrokCodeBook/code \em. Now the path to the newly created directory should appear before the prompt. The directory \em \~/Desktop/GrokCodeBook/code is now the current working directory.

Create a folder for the repository to reside in by issuing the command \em mkdir MessageN \em. Change directories into \em MessageN \em.

Issue the command \em git init \em to create the repository. There is now a Git repository at \em \~/Desktop/GrokCodeBook/code/MessageN \em.

Move the previously created build file to the \em MessageN \em directory using the command \em mv ../../default.build default.build \em.

Edit the build file so that it appears as follows:

\begin{tabbing}
$\langle$?xml version="1.0"?$\rangle$ \\
$\langle$pro\=ject name="MessageN" default="build" basedir="."$\rangle$ \\
\>$\langle$description$\rangle$A message server built in \CSharp.NET$\langle$/description$\rangle$ \\
\>$\langle$tar\=get name="build"$\rangle$ \\
\>\>$\langle$echo message="Building MessageN!"/$\rangle$ \\
\>$\langle$/target$\rangle$ \\
$\langle$/project$\rangle$
\end{tabbing}

Stage the build file for a commit to the repository from the Git Bash window with \em git add default.build \em. Commit the build file using \em git commit -m "Added build file" \em.

\section{Begin With A Test}

\subsection{Get NUnit}
Download NUnit from \url{http://www.nunit.org/index.php?p=download}, and run the installer.

Ensure that the directory containing \em nunit.exe \em appears in your \em PATH \em environment variable. Check this by issuing the command \em echo \%PATH\% \em from a Windows command line (won't work in msysgit/cygwin). Alternatively, execute the command \em nunit \em. This should launch the NUnit \gls{GUI} if the directory containing NUnit's binaries is in your \em PATH \em.

\appendix
% Appendix on Windows Administrative Tasks
%appendicies/windows-admin/windows-admin.tex

%TODO: find someone with amazing windows-foo to contribute to this section.

\chapter{Windows Administrative Tasks} 
\section{Setting the PATH Environment Variable}\label{path}
The \em PATH \em environment variable represents a semi-colon delimited (separated) list of all the paths in a computer where executable files are kept. This allows the execution of an executable from the command line by typing only it's name and not its full path (i.e. \em C:$\backslash$$\rangle$firefox \em as opposed to \em C:$\backslash$$\rangle$"Program Files$\backslash$Mozilla Firefox$\backslash$firefox.exe" \em).

The \em PATH \em environment variable may be set by launching the System Properties window. This can be done by pressing the \em Windows \em and \em Pause \em keys simultaneously. This key comibination is denoted by \em Windows  Pause \em (you will see this \em button + button \em shorthand for key combinations again). A dialog named \em System Properties \em will show up on-screen. Click the \em Advanced \em tab. Then click the \em Environment Variables \em button. If administrative priveliges are available, select the variable named \em PATH \em (spelling may also be \em Path \em) in the list called \em System variables \em, and push the \em Edit \em button.

The \em Edit System Variable \em dialog will display on screen. In the \em Variable value \em box, paste the path to the program (full path to the directory it is in) at the end of the list of paths. Ensure there is a semi-colon separating new path from the one before it. Click the \em OK \em button to close the \em Edit System Variable \em dialog. Click \em OK \em to close the \em Environment Variables \em dialog, and close the \em System Properties \em window.

\em N.B.: any open command windows while modifing your PATH variable will not reflect the changes to it. Be sure to close and reopen all command windows when editing the PATH. \em

\section{Launching a Command Window}\label{launch-command-window}
Launching a command window can be done by typing ``cmd'' into the \em Run \em box and pressing \em Enter \em. The \em Run \em box can be found from the \em start \em menu on Windows (look for the text ``Run...'' on the start menu). Alternatively, the \em Run \em box may be launched by the pressing \em Windows + R \em key combination on the keyboard.

\section{Changing Directories}\label{change-directories}
To change the current working directory (cwd) in a command window, issue the command ``cd'' followed by the name of directory to be made the new current working directory.

For example: \em C:$\backslash$Users$\backslash$GrokCodeBook$\rangle$ cd Desktop \em.

The above command would take you from the user's home directory (the user's profile directory---C:$\backslash$Users$\backslash$GrokCodeBook if you are the user ``GrokCodeBook'') to the Desktop in Vista or Windows 7.

\section{Create a Directory}\label{create-directory}
The ``mkdir'' command can be used to create a sub-directory from a directory if have sufficient permissions have been granted.

For example: \em C:$\backslash$Users$\backslash$GrokCodeBook$\backslash$Desktop$\rangle$ mkdir GrokCodeBook \em.

The above command would create the directory C:$\backslash$Users$\backslash$GrokCodeBook$\backslash$Desktop$\backslash$GrokCodeBook.

\section{Tab Completion}\label{tab-completion}
Tab completion functionality saves the user of the command window from having to type the full name of a directory when working with paths.

For example, when \hyperref[launch-command-window]{launching a command window}, the command window will default to the user's profile directory. To \hyperref[change-directories]{change directories} to the Desktop, one can type \em cd D \em then press the \em Tab \em key. This will complete the path. ``cd Desktop'' should be visible after the prompt if there are no other directories that begin with a ``D'' in the user's profile directory. 

If there are sub-directories with similar names (for instance ``Desktop'' and ``Downloads'') in the current working directory, the user need only type enough of the path (``De'' or ``Do'' in the case of ``Desktop'' and ``Downloads'') for the system to be able to discriminate between the two. 

If the system can't tell the difference (for example: only ``D'' was typed from the previous example), in Windows, typing the \em Tab \em key successive times will toggle between subdirectories with the same beginning (on *NIX systems, a list of sub-directories with the same beginning will be printed to the command window).

\printglossaries

\end{document}