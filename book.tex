\documentclass{book}

\usepackage[T1]{fontenc} % use the T1 font encoding
\usepackage{lmodern} % use computer modern font

\usepackage{hyperref}
\hypersetup{colorlinks=true,
	linkcolor=red} % use red colored links instead of weird boxes
	
\usepackage{verbatim} % allow verbatim input for things like code files http://www.tex.ac.uk/cgi-bin/texfaq2html?label=verbfile

% glossaries package needs to come after hyperref if glossary entries are to be clickable
\usepackage[toc,acronym]{glossaries}

%Terms

\newglossaryentry{auto-completion}
{
name=auto-completion,
description={``A feature [of software programs] that suggests text automatically based on the first few characters that a user types.''\footnote{http://msdn.microsoft.com/en-us/library/dd921717(v=office.12).aspx}}
}

\newglossaryentry{refactoring}
{
name=refactoring,
description={``The process of modifying source code for the purpose of improving its readability and maintainability while retaining the program's functionality and behavior.'' \footnote{\url{http://developer.apple.com/library/ios/\#documentation/DeveloperTools/Conceptual/Xcode_Glossary/20-Glossary/Glossary.html}}}
}

%Acronyms

%use lower case for glossary entry
%IDE
\newglossaryentry{ide}
{
name=\glslink{IDE}{Integrated Development Environment},
text=Integrated Development Environment,
description={``An integrated development environment (IDE) (also known as integrated design environment, integrated debugging environment or interactive development environment) is a software application that provides comprehensive facilities to computer programmers for software development''\footnote{\url{http://en.wikipedia.org/wiki/Integrated_development_environment}}}
}

%use all-caps for acronym and in body text
\newacronym[description={\glslink{ide}{Integrated Development Environment}}]{IDE}{IDE}{Integrated Development Environment}

%FLOSS
% use \- to suggest hyphenation points
\newglossaryentry{floss}
{
name=\glslink{FLOSS}{Free/Libre/Open Source Software},
text=Free\/\-Libre\/\-Open Source Software,
description={``Free/\-Libre/\-Open-\-source software (FLOSS) is liberally licensed to grant the right of users to use, study, change, and improve its design through the availability of its source code'' \footnote{\url{http://en.wikipedia.org/wiki/Free_and_open_source_software}}}
}

\newacronym[description={\glslink{floss}{Free/\-Libre/\-Open Source Software}}]{FLOSS}{FLOSS}{Free/Libre/Open Source Software}

%NAnt
\newglossaryentry{nant}
{
name=\glslink{NAnt}{NAnt},
text=NAnt,
description={A \gls{FLOSS} build automation tool written in C\#\footnote{\url{http://nant.sourceforge.net/}}}
}

\newacronym[description={\glslink{nant}{Short for ``Not Ant''\footnote{\url{http://nant.sourceforge.net/release/latest/help/introduction/fog0000000042.html}}}}]{NAnt}{NAnt}{Not Ant}

%TDD
\newglossaryentry{tdd}
{
name=\glslink{TDD}{Test Driven Development},
text=Test Driven Development,
description={A software development methodology whereby the developer writes a failing test, writes code to make the test pass, and refactors the test as needed to prevent duplication of code and brittleness. The tests are automated with the use of an automated test harness.}
}

\newacronym[description={\glslink{tdd}{Test Driven Development}]{TDD}{TDD}{Test Driven Development}

%VCS
\newglossaryentry{vcs}
{
name=\glslink{VCS}{Version Control System},
text=Version Control System,
description={A software tool for managing changes made to source code. Make \gls{branching} and \gls{merging} possible.}
}

\newacronym[description={\glslink{vcs}{Version Control System}]{VCS}{VCS}{Version Control System}

%DVCS
\newglossaryentry{dvcs}
{
name=\glslink{DVCS}{Distributed Version Control System},
text=Distributed Version Control System,
description={A distributed form of \gls{vcs}. Whole repositories exist on the developer's machine rather than on a centralized server. Changes are sent between developers rather than client to server.}
}

\newacronym[description={\glslink{dvcs}{Distributed Version Control System}]{DVCS}{DVCS}{Distributed Version Control System}

%CI
\newglossaryentry{vcs}
{
name=\glslink{CI}{Continuous Integration},
text=Continuous Integration,
description={An automated process whereby source code is built and tested continuously, and reports on the outcome of the build/test process are sent to the developers.}
}

\newacronym[description={\glslink{vcs}{Continuous Integration}]{CI}{CI}{Continuous Integration}

\makeglossaries

%defs.tex

\ifx \ifmonospace \undefined
	%http://www.opensource.apple.com/source/autoconf/autoconf-12/autoconf/config/texinfo.tex
	\def\ifmonospace{\ifdim\fontdimen3\font=0pt }
\fi

%http://stackoverflow.com/questions/2724760/how-to-write-c-in-latex/2728020#2728020
%C#
\def \CSharp {
\ifmonospace
    C\#
\else%
    %C\kern-.1667em\raise.30ex\hbox{\smaller{\#}}
    C\# %TODO: make this match example in url above
\fi
\spacefactor1000 }

\begin{document}

\begin{titlepage}

\begin{center}

{\LARGE Grok Code Book}

{\large \href{http://twitter.com/\#!/GrokCodeBook}{@GrokCodeBook}}

\end{center}
 
\vfill
 
{\small This work is licensed under the Creative Commons Attribution-NonCommercial-ShareAlike 3.0 Unported License. 
To view a copy of this license, visit \\
\url{http://creativecommons.org/licenses/by-nc-sa/3.0/} or send a letter to \\
Creative Commons, 444 Castro Street, Suite 900, Mountain View, CA, 94041, USA.}

\begin{center}

{\large \today}

\end{center}

\end{titlepage}



\newpage

\pagenumbering{roman}
\tableofcontents
\addcontentsline{toc}{chapter}{Contents}

% The star suppresses numbering
\chapter*{Introduction}\normalsize
\addcontentsline{toc}{chapter}{Introduction}
\pagestyle{plain}

The solitary nature of much of computer programming and the number of different avenues that may lead to a career in programming make it 
difficult for best practices to permiate the field. A responsible author can no longer assume that a person seeking to learn about programming 
has any background in either computer science or mathematics. All jargon should be introduced in a glossary at minimum; regardless of how many 
works the author has authored. If a programming text is to build on some assumed knowledge, it must explicitly define said knowledge from the 
outset, and ought contain a ``Read First'' section. 

Also, the volume of information forthcoming about new technologies---frameworks, toolkits, tooling, etc.---drowns out information about programming 
well. By providing this text free of charge, holding ourselves to the highest standard of authorship and editorship, trying to make this text 
accessible to all that might read it, and acting as stewards and care-takers of this living document, we hope to raise the bar for published 
works pertaining to computer programming.

The examples included hereafter are executed in \CSharp and Microsoft.Net, but the concepts set forth therein are relevant to all OO languages and frameworks.

% Need this here to force roman numbering for Intro and arabic for following chapter.
\newpage

\pagestyle{headings}
\pagenumbering{arabic}

% Setting Up A Programming Environment
%setup.tex

\chapter{Setting Up Your Programming Environment}

\section{DIY}
One click installers for all-in-one \glspl{IDE} can be nice, but they tend to encourage dependency on automata. They also promote, ``Well, if there 
isn't an installer for it, it can't be any good,'' thinking. It should be noted that often times creating an installer that works for Windows can be
prohibitively expensive for authors of tools who do not use Windows as their primary operating system or Visual Studio as their primary development 
environment.

\gls{FLOSS} software will be used in the examples hereafter; not because if its price-point, but rather because of the ability of open source software
to transfer knowledge to its end user. Want to know the best way to write a task for your build engine? Why not see how the authors of your build
engine did it? In order to write code, you must learn how to read code. By extension, to write code well, you must learn how to read code well. It is 
important to figure out why a piece of code was written the way it was and not just settle for duplicating and succesfully compiling it. Reading the 
code behind open source software is a cheap and easy way to learn from the masters. \gls{NAnt} contains some of the most well written \CSharp code out there.

\gls{FLOSS} software can be a little more difficult to set up, but doing so is worth the extra effort. There are many freely available tutorials
online that will walk you through how to configure a software package for first-time use. In the process of setting up software, you will learn a
little bit more about your computer runs programs in addition to making your computer a little bit more your own.

The biggest reason for configuring your own environment is personal taste. You may prefer typing text on the command line in a program like 
Vim\footnote{\url{http://www.vim.org/}} over using an \gls{IDE}.

When it comes to programmer tools, one size definitely does not fit all. Something like ReSharper\footnote{\url{http://www.jetbrains.com/resharper/}} 
may give you unparalelled \gls{refactoring} and \gls{auto-completion} support, but starting Visual Studio to change a configuration setting or one 
line of code is overkill. One theme you should expect to encouter repeatedly in the passages that follow is, 
``Use the right tool for the job at hand.''

\begin{quotation}
If all you have is a hammer, everything looks like a nail.

---Bernard Baruch
\end{quotation}

Equip yourself appropriately.

\section{Tooling}

\subsection{Build Automation}

You should always provide other developers on your team the ability to quickly build your code and test it (with automated tests as well as 
manually). Testing should always be a part of the review process as exercising the code can and does reveal errors that a visual inspection may 
miss. Build automation is the mechanism by which you provide other developers this ability.

Tasks within your build automation system can be used to check style, run automated tests, and write reports as well as compiling your code.

Build automation is a critical to the practice of \gls{TDD}. If tests cannot be easily automated, they will not be run. Not out of developer laziness,
but for the reason of sheer number. As your codebase grows, your tests can easily number in the hundreds. If you don't have a test harness for
automating those tests, they simply won't be run often enough.

\gls{NAnt} will be used in the examples that follow. MSBuild has some advantages in Windows-only shops---i.e. it is
installed by default on Windows Server machines. At the time of writing, it more of a hassle to run all the unit tests for a project from the command
-line in MSBuild than in NAnt. Also, there seems to be more of an air of acceptance towards customization---specifically writing one's own build 
tasks---in the NAnt community. 

While there is virtue in the philosophy, ``Why write what you can download for free?'' sometimes the needs of 
customization outweigh the benefits of using easily downloadable/purchased software. The philosophy set forth herein is, ``Weigh all options without bias, and make the
logical choice.'' Developers are---or at least always should be---part of a team. Sometimes, for the sake of progress, a developer needs to make
decisions that compromise on their own happiness. For instance, not using your favorite build tool on a fresh project in a new job, because all the
other developers are invested in the use of another tool.

\subsection{Test Driven Development}

\gls{TDD} should be thought of as a developer's safety net. Trapeze artists use a safety net to keep them from falling to their deaths. The presence
of that net allows them to attempt things they might not otherwise. It allows them to overcome their fear to achieve greatness. \gls{TDD} is first 
and foremost about giving developers the confidence to improve their code.

\gls{NUnit} will be used in later examples as more support exists to automate it from the client machine.

\subsection{Version Control}
Things don't always go as planned. Sometimes defects will evade detection in testing. Having a way to quickly revert to a known working state is a 
luxury no developer can afford to go without. Version control can be a lot like insurance to the uninitiated: you don't know you need it until it is 
too late.

\glspl{VCS} also provide a facility called branching. \Gls{branching} permits experimentation within libraries. \Gls{merging} allows succesful 
experiments be kept. Otherwise, the branch can simply be abandoned. 

%TODO: find someone who knows something about this to write this bit. Dropbox?
\subsection{Backup}
To prevent lost work, source code and application configuration should be backed up regularly. \glspl{DVCS} have the advantage that each developer 
has a copy of the entire repository. Thus, as changes are passed back and forth between developers, backup occurs organically.

\subsection{Continuous Integration}
Based on the philosophy of making the hard things to do easier by doing them more often, ideally, continuous integration provides a means for 
keeping software in a ready-to-deploy state.

\gls{CI} systems wait for a trigger, and then perform a series of tasks once that trigger has been detected such as running unit tests,
emailing reports, updating documentation, and even performing version control tasks.

Common triggers are changes to the source code repository, elapsed time, and forced builds.

The \gls{CI} server has oft been called the heartbeat of the software project.

In the following pages, \gls{Buildbot} will be used as the CI server.

\section{A First Build}

\subsection{Get NAnt}
Download \gls{NAnt} from \url{http://sourceforge.net/projects/nant/files/nant/}. As of the time of this writing, it is recommended that you get
either the stable release of 0.90 or the nightly build. Unzip to C:$\backslash$Program Files$\backslash$nant-$\langle$version$\rangle$ or another centrally available 
location.

Modify your \em \hyperref[path]{\%PATH\%} \em environment variable to point to $\langle$NAnt-Installation-Directory$\rangle$$\backslash$bin (i.e. C:$\backslash$Program Files$\backslash$nant-0.90$\backslash$bin).

Test NAnt by \hyperref[launch-command-window]{launching a command window} and typing \em nant \em at the prompt.

\subsection{Create Your First Build File}
With your open command window, \hyperref[change-directories]{change directories} to your desktop.

\hyperref[create-directory]{Create a directory} named ``GrokCodeBook'' using the ``mkdir'' command.

Open a file for editing with your favorite \gls{text-editor} (a \hyperref[text-editors]{listing of text-editors} is provided in the \hyperref[tools]{Tools} appendix) and type the following:

%http://kb.mit.edu/confluence/display/ist/How+to+use+Tabs+in+LaTeX
\begin{tabbing}
$\langle$?xml version="1.0"?$\rangle$ \\
$\langle$pro\=ject name="Hello Grok Code Book" default="msg" basedir="."$\rangle$ \\
\>$\langle$description$\rangle$An example build file$\langle$/description$\rangle$ \\
\>$\langle$tar\=get name="msg"$\rangle$ \\
\>\>$\langle$echo message="Hello!"/$\rangle$ \\
\>$\langle$/target$\rangle$ \\
$\langle$/project$\rangle$
\end{tabbing}

Save the file as $\langle$Desktop$\rangle$$\backslash$GrokCodeBook$\backslash$default.build.

% Keeping It Classy
% classy.tex

\chapter{Keeping It Classy}

Countless examples of structured programing in an object oriented language presented as proper technique exist both online and in programming texts. This is the equivalent of buying a set of golf clubs, but only ever using the seven iron.

Object orientation allows for the separation of concerns and facilitates refactoring. Two methodologies that allow changes to be introduced to software systems relatively easily.

Going forward, \href{http://www.javaworld.com/javaworld/jw-08-1999/jw-08-interfaces.html}{coding to interfaces} will be presented as the appropriate way to implement object oriented systems. This approach allows for the development of loosely coupled and highly cohesive systems. Coding for interfaces also allows the developer to prioritize development by factors like risk mitigation rather than requiring a class to wait on all it's components to be concretely implemented before starting work on that class. Coding to interfaces also allows for division of labor and collaboration. Perhaps most importantly, coding to interfaces allows objects to be tested in isolation.



% Configuration
% chapters/config/config.tex

\chapter{Extensibility Through Configuration}

Extensibility is a \href{http://webarya.wordpress.com/2010/05/28/my-ten-development-principles/}{delicate subject}. \href{http://c2.com/xp/BigDesignUpFront.html}{Big Design Up Front} is a perjorative term coined to describe the fool-hardy attempt to completely specify a software system in full detail before writing a line of code. This approach strives to prevent waste by not working on the problem until it has been completely defined. The problem is that the only absolute in software is uncertainty. Implied requirements usually can't be discovered until a customer has seen the software in action and has the opportunity to say, ``That isn't what I meant.'' 

Designing for extensibility where it isn't required or isn't high-priority can easily lead to \href{http://blog.digitalstruct.com/2008/02/17/over-engineering-software/}{over-engineering the software}. The ways to prevent over-engineering are to deliver working tested software ASAP rather than a huge feature set (contracts can be extended later if need be and only successful projects deserve extension), to think and reason, and to continually check priorities with the customer (communicate as much as possible with as much efficiency as possible).

Preferring composition (using components that perform specific functions such as sending a message) over inheritence is one development strategy that aims to reduce over-engineering and promote possible future re-use by avoiding unnecessary abstraction. Creating base classes is an activity few people do well. Following the \href{http://www.objectmentor.com/resources/articles/srp.pdf}{Single Responsibility Principle} can also help keep developers focused on just the task at hand.

From discussions with the customer for the messaging service, it can be determined that extensibility is a priority in at least one case: the customer would like to be able to schedule both email messages and Facebook updates. Scheduling Facebook messages has also been prioritized over scheduling email.

As a starting point, here's a failing test for a list of transports retrieved from configuration:

\verbatiminput{chapters/config/document_assets/GrokCodeBook/code/MessageN/projects/MessageN.Test/MessageTransportConfigurationSectionTest-01.cs}

Running the build file from the \em MessageN.Test \em directory, should report a message indicating that the \em Should\_Return\_List\_Of\_Transports\_From\_Transports \em test failed.

This test runs, but it does not communicate anything about the behavior of the \em MessageTransportConfigurationSection \em.

To this end, modify \em MessageTransportConfigurationSectionTest \em as follows:

\verbatiminput{chapters/config/document_assets/GrokCodeBook/code/MessageN/projects/MessageN.Test/MessageTransportConfigurationSectionTest-02.cs}

A MessageTransportConfigurationSection class needs to be defined in a new \em Configuration \em subdirectory under \em Desktop$\backslash$GrokCodeBook$\backslash$code$\backslash$MessageN$\backslash$projects$\backslash$MessageN \em so that this class can compile. Name the class file \em MessageTransportConfigurationSection.cs \em, and add the following text to it:

\verbatiminput{chapters/config/document_assets/GrokCodeBook/code/MessageN/projects/MessageN/Configuration/MessageTransportConfigurationSection-01.cs}

This modification will allow the test to compile, but it will still fail.

Alter the test to add a transport to the section:

\verbatiminput{chapters/config/document_assets/GrokCodeBook/code/MessageN/projects/MessageN.Test/MessageTransportConfigurationSectionTest-03.cs}

And make a matching edit in the section so the test will compile and run:

\verbatiminput{chapters/config/document_assets/GrokCodeBook/code/MessageN/projects/MessageN/Configuration/MessageTransportConfigurationSection-02.cs}

Now the section has a \em TransportCount \em rather than a \em Transports \em property exposed. Exposing the list of transports was breaking encapsulation and had a bit of a \href{http://c2.com/cgi/wiki?FeatureEnvySmell}{Property Envy Smell} to it.

\appendix
% Appendix on Windows Administrative Tasks
%windows-admin.tex

\chapter{Windows Administrative Tasks}
\section{Setting the PATH Environment Variable}
Do this by launching the System Properties window. This can be done by pressing the Windows and Pause keys simultaneously. This key comibination is
denoted by ``Windows + Pause'' \(you will see this shorthand for key combinations again\).

% Appendix on Tools
%appendicies/tools/tools.tex

\chapter{Tools}\label{tools}
\section{Text-editors}\label{text-editors}
\subsection{CLI}
\begin{itemize}
\item gvim
\item edit (ships with Windows)  
\end{itemize}

\subsection{GUI}
\begin{itemize}
\item Notepad++
\item GVim
\item notepad.exe (ships with Windows)
\end{itemize}  


% Appendix on nRake
% appendicies/nrake/nrake.tex

\chapter{nRake}\label{nrake}
nRake is an open source build automation tool written in Ruby in the Rake build language. nRake can use either IronRuby or Ruby 1.8/9 implementations.

The following example builds the \em MessageN \em and \em MessageN.Test \em libraries and runs the tests contained in \em MessageN.Test.dll \em.

\verbatiminput{appendicies/nrake/document_assets/GrokCodeBook/Rakefile}

It should reside in the root of your folder structure (i.e. \em Desktop$\backslash$GrokCodeBook \em).

\printglossaries

\end{document}